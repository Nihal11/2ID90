\documentclass[11pt,a4paper]{article}
\usepackage[latin1]{inputenc}
\usepackage[english]{babel}
\usepackage{amsmath}
\usepackage{amsfonts}
\usepackage{amssymb}
\usepackage{graphicx}
\usepackage[margin=1in]{geometry}
\usepackage[linesnumbered,noend]{algorithm2e}
\usepackage{tabularx}
\usepackage{hyperref}
\usepackage{xcolor}
\hypersetup{ % Copied from: http://tex.stackexchange.com/a/847/41003
    colorlinks,
    linkcolor={red!50!black},
    citecolor={blue!50!black},
    urlcolor={blue!80!black}
}

% New paragraph = blank line, not indent.
\setlength{\parskip}{0.3cm}
\setlength{\parindent}{0pt}

% Code listings
\usepackage{listings}

\title{An evaluation function for a minimax Draughts player}
\author{
Dennis van der Schagt (0814249)\\
Rob Wu (0787817)
}
\date{\today}


\begin{document}
\maketitle
\newpage
\section{The algorithm}

\begin{function}[H]
	\DontPrintSemicolon
	\caption{alphabeta(node, remainingDepth, $\alpha$, $\beta$)}
	\KwIn{node, remainingDepth, $\alpha$, $\beta$}
	\If{a player won in node or $remainingDepth = 0$}{
    	\Return{the heuristic value of node}
    }
    \eIf{it's white's turn in node}{
    	\For{each child of node}{
    		$\alpha$:=  maximum of $\alpha$ and alphabeta(child, remainingDepth - 1, $\alpha$, $\beta$)\\
            \If{$beta \leq alpha$}{
				\Return{$\beta$} \tcc*{Beta cut-off}
            }
		}
        \Return{$\alpha$}
    }{ % else
    	\For{each child of node}{
    		$\beta$:= minimum of $\beta$ and alphabeta(child, remainingDepth - 1, $\alpha$, $\beta$)\\
			\If{$\alpha \leq \beta$}{
				\Return{$\alpha$} \tcc*{Alpha cut-off}
			}
		}
		\Return{$\beta$}
	}
\end{function}

\section{Evaluation function}
We intentionally kept our evaluation function simple and fast. Keeping it simple makes it easier to understand what is going on and thus easier to improve the magic numbers. Because of the simple evaluation function we also had an easier time fixing a bug we had in our alpha-beta algorithm. Keeping the function fast allows us to look forward as many turns as possible in the given time.

The evaluation function returns a single integer value which reflects the state of the game. A negative value means that black is likely going to win, whereas a positive value means that white will probably win. When the game state is such that one of the players is certainly going to win, the returned value is an extreme value. E.g. if black will certainly win from white, then the returned value is the maximum representable negative integer.

If it is not set in stone which player will win, then the score is calculated from the number of pieces on the board and their position.
We assume that the number of pieces is the most important indicator of success, so this feature is the primary score component. To make sure that this primary choice is always respected irrespective of the position scores, the scores are chosen such that the sum of the position scores can never exceed the smallest unit of the piece count score.

The first score modifier is used to value the pieces that are on the board. For this part we only look at the kind of piece, white or black, uncrowned or king. We give a value of $1000$ to a white uncrowned piece. On advice of dr.ir. J.W. Wesselink we decided to value kings thrice as high as an uncrowned piece, so a white king is worth $3000$. This makes sure our player will work toward getting a king without giving up too many uncrowned pieces for that goal. The values of the black pieces are the opposite of the values of the white pieces, as we want to have a symmetric evaluation function.

The second score modifier associates a value to a piece depending on their row position. The closer a piece is to the other side of the board, the higher its bonus score. This criteria exists to encourage reaching the other end in order to get a king, even when the AI agent cannot look that deep in the game tree. Because this criteria only exists to stimulate becoming a king, the bonus is only added to uncrowned pieces. Note that the piece count takes precedence over position, so the agent will not blindly offer pieces in an attempt to get a piece to the other end, because this action would negatively affect the piece count.\\
To avoid moving too many pieces to the other side of the board, the first row will get a higher bonus score than the first few rows, to encourage pieces to stay home. This allows them to prevent opposing pieces from being a king.
\\
A way to implement this bonus system is by awarding one bonus point to a piece for moving one row forwards. A disadvantage of this mechanism is that it does not differentiate between a piece that is already close to the end, and a piece that is still at first rows. To distinguish between both situations, the list of bonus scores for each row is a Fibonacci sequence. The final list of bonus scores is shown in table \ref{table:bonusscore}. There are at most 20 pieces per player. The maximum score for a piece is $31$, so the maximum bonus is not higher than $20\times 31 = 620$, which is well below $1000$ (the piece count score for an uncrowned piece).

\begin{table}[ht]
\begin{tabularx}{\linewidth}{l|l|X}
row & score & comment \\
\hline
1 & 10 & Staying home is preferable to moving one forward, in order to defend against incoming pieces from the opponent. \\
2 & 1 & \\
3 & 2 & = more than 1 \\
4 & 3 & = 1 + 2\\
5 & 5 & = 2 + 3\\
6 & 8 & $\dots$\\
7 & 13 & \\
8 & 21 & \\
9 & 31 & \\
10 & 0 & A piece at the end is already a king. There's no need to associate a score with a piece at the last row, because this case does not occur. \\
\end{tabularx}
\caption{Bonus scores for piece position.}\label{table:bonusscore}
\end{table}

\section{Results + evaluation}
% Mogelijke vragen die we hier kunnen beantwoorden
What is our player good at?\\
What could we have improved?\\
What kind of players do we win from?\\
To what extend does the search horizon play a role? % Kan misschien naar section `algorithm` verplaatst worden

\section{References}
\url{http://www.cs.columbia.edu/~devans/TIC/AB.html}
%\url{http://stackoverflow.com/a/15626976/1928529} of \url{http://en.wikipedia.org/wiki/Alpha%E2%80%93beta_pruning#Pseudocode} (Voor alphabeta pseudocode)
\section{Contributions}
\section{Reproducing the results}


\end{document}














