\documentclass[11pt,a4paper]{article}
\usepackage[latin1]{inputenc}
\usepackage[english]{babel}
\usepackage{amsmath}
\usepackage{amsfonts}
\usepackage{amssymb}
\usepackage{graphicx}
\usepackage[margin=1in]{geometry}
\usepackage[linesnumbered,noend]{algorithm2e}
\usepackage{hyperref}
\usepackage{xcolor}
\hypersetup{ % Copied from: http://tex.stackexchange.com/a/847/41003
    colorlinks,
    linkcolor={red!50!black},
    citecolor={blue!50!black},
    urlcolor={blue!80!black}
}

% New paragraph = blank line, not indent.
\setlength{\parskip}{0.3cm}
\setlength{\parindent}{0pt}

% Code listings
\usepackage{listings}

\title{An evaluation function for a minimax Draughts player}
\author{
Dennis van der Schagt (0814249)\\
Rob Wu (0787817)
}
\date{\today}


\begin{document}
\maketitle
\newpage
\section{The algorithm}

\begin{function}[H]
	\DontPrintSemicolon
	\caption{alphabeta(node, remainingDepth, $\alpha$, $\beta$)}
	\KwIn{node, remainingDepth, $\alpha$, $\beta$}
	\If{a player won in node or $remainingDepth = 0$}{
    	\Return{the heuristic value of node}
    }
    \eIf{it's white's turn in node}{
    	\For{each child of node}{
    		$\alpha$:=  maximum of $\alpha$ and alphabeta(child, remainingDepth - 1, $\alpha$, $\beta$)\\
            \If{$beta \leq alpha$}{
				\Return{$\beta$} \tcc*{Beta cut-off}
            }
		}
        \Return{$\alpha$}
    }{ % else
    	\For{each child of node}{
    		$\beta$:= minimum of $\beta$ and alphabeta(child, remainingDepth - 1, $\alpha$, $\beta$)\\
			\If{$\alpha \leq \beta$}{
				\Return{$\alpha$} \tcc*{Alpha cut-off}
			}
		}
		\Return{$\beta$}
	}
\end{function}

\section{Evaluation function}
We intentionally kept our evaluation function simple and fast. Keeping it simple makes it easier to understand what is going on and thus easier to improve the magic numbers. Because of the simple evaluation function we also had an easier time fixing a bug we had in our alpha-beta algorithm. Keeping the function fast allows us to look forward as many turns as possible in the given time. Our evaluation function is build as a combination of three score modifiers.

The first and most important score modifier is activated when the currently evaluated node is a final game state in which someone won. If white won we return as score the maximum representable integer value. On the other hand, if black won, we return the minimum representable integer value. If someone won the other modifiers are not incorporated in the score as we value every win the same.

The second score modifier is used to value the pieces that are on the board.

The third, and least influential, score modifier gives a value to all pieces depending on how far they are from the back row.

\section{Results + evaluation}
% Mogelijke vragen die we hier kunnen beantwoorden
What is our player good at?\\
What could we have improved?\\
What kind of players do we win from?\\
To what extend does the search horizon play a role? % Kan misschien naar section `algorithm` verplaatst worden

\section{References}
\url{http://www.cs.columbia.edu/~devans/TIC/AB.html}
%\url{http://stackoverflow.com/a/15626976/1928529} of \url{http://en.wikipedia.org/wiki/Alpha%E2%80%93beta_pruning#Pseudocode} (Voor alphabeta pseudocode)
\section{Contributions}
\section{Reproducing the results}


\end{document}














