\documentclass[11pt,a4paper]{article}
\usepackage[latin1]{inputenc}
\usepackage[english]{babel}
\usepackage{amsmath}
\usepackage{amsfonts}
\usepackage{amssymb}
\usepackage{graphicx}
\usepackage[margin=1in]{geometry}
\usepackage[linesnumbered,noend]{algorithm2e}
\usepackage{tabularx}
\usepackage{hyperref}
\usepackage{xcolor}
\hypersetup{ % Copied from: http://tex.stackexchange.com/a/847/41003
    colorlinks,
    linkcolor={red!50!black},
    citecolor={blue!50!black},
    urlcolor={blue!80!black}
}

% New paragraph = blank line, not indent.
\setlength{\parskip}{0.3cm}
\setlength{\parindent}{0pt}

% Code listings
\usepackage{listings}

\title{Spelling Correction}
\author{
Dennis van der Schagt (0814249)\\
Rob Wu (0787817)
}
\date{\today}


\begin{document}
\maketitle
\newpage
\section*{Abstract}


\section{The algorithm}
To find the best suggestion for a spelling correction, potential spelling corrections need to be generated and evaluated.

We assume that the input text contains at most two misspelled words, which are not next to each other.
To find the most appropriate spelling correction, we first rate the original text, and then re-evaluate the text with one or two words modified.
We apply the noisy channel model and assume that a word contains only one typographical mistake (insertion, deletion or substitution of one character, or transposition of two consecutive characters). For each input word, a list of such corrections is generated. These corrections are called \textit{candidates}. The output where words may be replaced by candidates is called a \textit{suggestion}.

To evaluate a suggestion, the likelihood of a candidate is used, as well as the likelihood of the bigrams of candidates within the suggestion.
In the next sections, these are explained in detail. The suggestion where the combined likelihoods is maximized is used.

%TODO Improve pseudo-code.
\begin{lstlisting}
// Split the input sentence into words.

// Pre-calculate channel model probabilities.

// Calculate the initial probability of the input word.

// Generate plausible suggestions for spelling correction,
// and evaluate their accuracy.

// Present the final suggestion.
\end{lstlisting}

\subsection{Channel model probability}
\subsection{Bigram language model probability}
\subsection{Generating suggestions}

\section{Results}

\section{Reproducing the results}
Following are the steps required to reproduce our results:
\begin{itemize}
\item Download the Netbeans project of our Spelling correction program from Peach.
\item Compile the project, e.g. using \textsc{ant compile}.
\item Run the project, e.g. using \textsc{ant run}.
\item Type a sentence consisting of lowercase Latin letters and spaces, with at most two spelling errors and press enter.
\item Read the result. The corrected sentence will be displayed as "Answer: [corrected sentence].
\end{itemize}

\section{References}


\section{Contributions}
Dennis and Rob contributed equally to this assignment.


\end{document}














